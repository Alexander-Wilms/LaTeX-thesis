\documentclass[a4paper,12pt]{article}
% Umlaute im PDF sind nicht zusammengesetzt
\usepackage[T1]{fontenc}
% Umlaute müssen nicht maskiert werden
\usepackage[utf8]{inputenc}
% Titel
\title{Simulation und Realisierung der Gelenkregelung G2 Stäubli TX60L}
% Autor
\author{Fabian Alexander Wilms, Salvador Tazanou}
% Vektorschriftart
\usepackage{lmodern}
% für Deutsche Begriffe und Silbentrennung
\usepackage[ngerman]{babel}
% Formeln
\usepackage{amsmath}
% Grafiken
\usepackage{graphicx}
% Abkürzungsverzeichnis
\usepackage[printonlyused]{acronym}
% Stichwortverzeichnis
\usepackage{makeidx}
\makeindex
% idxlayout works around a bug in LaTeX, which results in a wrong vertical position of the title
% http://tex.stackexchange.com/questions/23287/setting-the-same-distance-from-the-top-of-the-page-
% for-chapter-and-index-titles
\usepackage{idxlayout}
% Zeilenabstand 1,5 Zeilen
\usepackage{setspace}
\onehalfspacing
% BibLaTeX benutzen
\usepackage[backend=biber,citestyle=alphabetic,bibstyle=alphabetic]{biblatex}
\addbibresource{bib/Bibliographie.bib}
% Formatierung an Vorgaben anpassen
\DeclareLabelalphaTemplate{
	\labelelement{
		\field[final]{shorthand}
		\field{label}
		\field[strwidth=3,strside=left,ifnames=1]{labelname}
		\field[strwidth=1,strside=left]{labelname}
	}
	\labelelement{
	\literal{-}
	}
	\labelelement{
	\field[strwidth=2,strside=right]{year}
	}
}
% Seitenränder anpassen
\usepackage{geometry}
\geometry{left=2cm, right=2cm, top=2.5cm, bottom=2cm}
% eps-Grafiken
\usepackage{epstopdf}
% Siehe http://tex.stackexchange.com/questions/28198/using-ieeetrantools
\usepackage[retainorgcmds]{IEEEtrantools}
\usepackage{microtype}
\usepackage{blindtext}
\newcounter{savepage}
% pdf-Metadaten und Hyperlinks
\usepackage[pdfsubject = {{}},
			pdfkeywords = {{Keyword Keyword2}},
			pdfstartview = Fit,
			pdfpagelayout = SinglePage,
			hidelinks,
			pdfusetitle]{hyperref}
% Schneide automatisch alle mit /includegraphics eingebundene PDFs zu
\usepackage{xstring}
\let\oldincludegraphics\includegraphics
\renewcommand{\includegraphics}[2][width=\textwidth]{%
	\immediate\write18{pdfcrop #2}%
	\StrSubstitute{#2}{.pdf}{-crop.pdf}[\temp]%
	\oldincludegraphics[#1]{\temp}%
	}
% Deutsche Anführungszeichen
\usepackage{csquotes}
\begin{document}
\pagenumbering{Roman}
\newpage\null\thispagestyle{empty}\newpage
\newpage
\pagestyle{empty}
\begin{center}
\includegraphics[width=0.35\textwidth]{fig/Logo/h-da-fbmk-logo-sw.pdf}
\vfill
\Large Hochschule Darmstadt \\
\vspace{12pt}
Fachbereich Maschinenbau und Kunststofftechnik \normalsize \\
\vfill
% \@title und \@author verfügbar machen
\makeatletter
\@title \\
\vfill
Dokumentation zum Seminar der Robotik \\
\vfill
vorgelegt von \\
\vspace{12pt}
\@author
\makeatother
\vfill
\begin{tabular}[h]{p{4cm}r}
	Betreuung: & A. König, W. Weber\\
\end{tabular}
\end{center}
\newpage
\phantomsection
\addcontentsline{toc}{section}{Inhaltsverzeichnis}
\tableofcontents
% Letzte Seite mit römischer Seitenzahl speichern
\cleardoublepage
\setcounter{savepage}{\arabic{page}}
\newpage
\cleardoublepage
\pagenumbering{arabic}
% Struktur basiert auf http://kleinmann.eit.h-da.de/99_ThesisInfos/
\section{Einführung}
Das Zentrum für Robotik hat den Industrieroboter TX60L mit der speziellen Schnittstelle \ac{LLI} beschafft. Mit dieser Schnittstelle ist es für die h\_da möglich, eigene Steuerungen und Regelungen zu realisieren. Kinematische Parameter und Parameter der Dynamik wurden ermittelt bzw. identifiziert\cite{D.X.Nodem.2015}. Für das Gelenk 1 wurde eine Einzelgelenkregelung realisiert (s. auch Versuch Labor Regelung von Roboterarmen). Ein Greifer wurde beschafft, der in das Robotersystem integriert werden kann.
\subsection{Was ist die Aufgabe?}
Für Gelenk 2 ist eine dezentrale Lageregelung in Kaskadenform in Anlehnung an die schon vorhandene Regelung für Gelenk 1 zu entwerfen, zu simulieren und zu realisieren. Dabei ist auch eine Eigengewichtskompensation vorzusehen und das Regelungsverhalten bei Lastmassenabwurf zu testen.
\subsection{Abzusehende Teilaufgaben}
\begin{itemize}
	\item Einarbeitung in Modellbeschreibung
	\item Erstellung des Simulationsmodells und Tests bei geeigneten Bewegungsvorgaben
	\item Einarbeitung in die Steuerungsumgebung mit LLI-Schnittstelle
	\item Montage des Greifers und Integration der Greiferansteuerung in die Steuerungsumgebung
	\item Implementierung der Steuerungsalgorithmen
	\item Experimentdurchführung und Vergleich mit der Simulation
	\item Dokumentation
\end{itemize}
\subsection{Was ist das Ziel dieser Arbeit?}

\subsection{Was war der Status Quo, bevor mit dieser Arbeit begonnen wurde?}

\subsubsection{Firma X hat etwas in der Vergangenheit auf diesem oder jenem Wege gemacht…}

\section{Problemlösung}

\subsection{Was sind die Optionen / Welche prinzipiellen Lösungen sind möglich?}

\subsection{Stand der Technik / Wie wird es woanders gemacht?}

\subsection{Was ist der in dieser Arbeit gewählte Ansatz?}

\subsection{Weshalb wurde dieser Ansatz gewählt / Was unterscheidet ihn von anderen / Was ist neu / Was ist bekannt?}

\subsection{Detaillierte Beschreibung der Problemlösung}

\section{Implementierung und Test}

\subsection{Wie wurde es implementiert}

\subsection{Wie wurde getestet}

\subsection{Warum wurde so getestet?}

\section{Validierung}

\subsection{Was sind die Ergebnisse}

\subsection{Vor- und Nachteile des entwickelten Systems}

\section{Zusammenfassung / Fazit / zukünftige Arbeit}

\subsection{Was wurde getan}

\subsection{Was muss noch getan werden / Wie kann das System verbessert werden?}

\newpage
\cleardoublepage
\pagenumbering{Roman}
% Seitenzahl wiederherstellen
\setcounter{page}{\thesavepage}
\clearpage
\phantomsection
\addcontentsline{toc}{section}{Literatur}
\printbibliography
\newpage
\renewcommand{\indexname}{Stichwortverzeichnis}
\printindex
\addcontentsline{toc}{section}{Stichwortverzeichnis}
\newpage
\section*{Abkürzungsverzeichnis}
\addcontentsline{toc}{section}{Abkürzungsverzeichnis}
\begin{acronym}[Bash] % längste Abkürzung
	\acro{LLI}{Low Level Interface}
\end{acronym}
\newpage
\phantomsection
\addcontentsline{toc}{section}{Abbildungsverzeichnis}
\listoffigures
\newpage
\phantomsection
\addcontentsline{toc}{section}{Tabellenverzeichnis}
\listoftables
\newpage
\section*{Anhang}
\addcontentsline{toc}{section}{Anhang}
\subsection{Protokoll}
\begin{itemize}
	\item 7.4.2016
	\begin{itemize}
		\item Projekt zugeteilt
		\item erste Unterlagen erhalten
		\item Vorschlag von Kommilitone, der zuvor bei Reis arbeitete: Quaderförmige Greifbacken mit gummierten Oberflächen
	\end{itemize}
	\item 8.4.2016
	\begin{itemize}
		\item Überlegungen zum Greifer: evtl. konkave Greifbacken
		\item Frage: Welche Form hat die Lastmasse?
		\item Wiedereinlesen in Kaskadenregelung
		\item Regler vereinfacht einen Teil der Strecke		
		\item Informieren über LLI
		\item Low Level Interface
		\item LLI gibt Zugriff auf Servos (Position control, Drehmoment control), IO und Vorwärtskinematik und Inverse Kinematik Solver
		\item Programmierung in C / C++
		\item Code für Motion Control \& Bahnplanung muss selber geschrieben werden
		\item Hardwarenah
		\item gut für Forschungszwecke (Universitäten und Forschungsinstitute)
		\item mit Dokumentation begonnen
	\end{itemize}
	\item 14.4
	\begin{itemize}
		\item wöchentliches Treffen mit Dr. Weber
		\item quaderförmiges Gewicht
		\item Handskizze reicht, muss keine CAD-Zeichnung sein
		\item der Greifer ist zunächst weniger wichtig, erst mal auf Simulink-Modell konzentrieren
		\item CDs mit Informationen zum LLI erhalten
	\end{itemize}
	\item 20.4
	\begin{itemize}
		\item Weiter am Simulink-Modell gearbeitet
	\end{itemize}
	\item 21.4.16
	\begin{itemize}
		\item wöchentliches Treffen mit Dr. Weber
		\item $R_{H0}$ zu bestimmen ist kompliziert, da immer auch die Gewichtskraft mit wirkt $\rightarrow$ Bei vollständig ausgestrecktem Arm testen
		\item Der PI-Regler sollte mit ins Simulink-Modell integriert werden
		\item Wir nehmen an, dass die Haftreibung konstant ist, auch wenn noch ein liearer Anteil an Gleitreibung dazukommt
		\item $\tau$ kann mithilfe des LLI vorgegeben werden, es muss nicht mit $U_S$ gerechnet werden
		\item die 6 kg Lastmasse sollen mit ins Modell des Gelenks genommen werden (So wurde es auch bei Gelenk 1 gemacht)
		\item Aufgaben, die zum nächsten Treffen erledigt werden sollen
			\begin{itemize}
				\item Regler in Simulink aufbauen, nicht in Matlab-Code
				\item Das Eigengewicht mit ins Modell übernehmen
				\item das Modell auf $\tau$ statt $U_S$ umstellen
				\begin{itemize}
					\item Die Sterne in den Symbolen $M^*$, $F_D^*$ und $F_M^*$ bedeuten, dass man mit $U_S$ rechnet. Das bedeutet, dass in sämtlichen Größen V enthalten ist. Eigentlich sollten diese sich nur auf das Moment beziehen.
				\end{itemize}
			\end{itemize}
		\item Die Beiblätter der zugeschickten Unterlagen gehören nicht zum Inhalt des Seminars
		\item Mit unserer Zeichnung des Greifers sollen wir nicht direkt zur Werkstatt gehen, sondern diese Herrn König geben
		\item Greifer:
		\begin{itemize}
			\item Schrauben werden von oben eingeschraubt
			\item die zwei inneren Schrauben je Greifer reichen aus
			\item im geschlossenen Zustand sollen die Greiferfinger einen Abstand von 3 cm haben
		\end{itemize}
	\end{itemize}
\end{itemize}
\end{document}