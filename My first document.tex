% LaTeX commands start with a slash, followed by the command name.
% Optional parameters are given between square brackets, obligatory
% parameters between curly braces. Many common commands are easily
% accessible in the toolbar and the menu above, as well as the sidebar
% on the left. You can enable the sidebar via the 'View' menu.

% Specify the type of document and its properties
\documentclass[a4paper,11pt]{article}

% Start of preamble
% Include packages which provide additional functionality
% In order to use some LaTeX-packages, you will need to install
% them via your package manager. The package names will likely
% differ, though.
\usepackage[T1]{fontenc}
\usepackage[utf8]{inputenc}
\usepackage{lmodern}
% required for equations
\usepackage{amsmath}
% required to include graphics
\usepackage{graphicx}
\title{My first \LaTeX{} document}
% Name of the author
\author{Jon Doe}
% End of preamble

% The entire document's contents are enclosed by the two document commands
\begin{document}
% Insert the title specified above
\maketitle
% Insert an auto-generated table of contents
\tableofcontents
\newpage

% The document structure is determined by the section and paragraph commands
% In order to easily navigate within your document, use the overview in the sidebar
\section{Introduction}
Welcome to \LaTeX ila. This document\footnote{And this is a footnote} is just meant to allow you to get
familiar with the general structure of a Latex document. To get a more in-depths understanding of LaTeX, you may want to read \emph{The Not So Short Introduction to \LaTeX{} 2$_\epsilon$}.

\subsection{Equations}
% Numerated equations are written like this
\begin{equation}
    E=m \cdot c^2
\end{equation}

\subsection{Graphics}
% Insert a numbered figure. The "h" option specifies that it won't be moved somewhere else
\begin{figure}[h]
    % Center the image
    \begin{center}
        % Include the graphics file and resize it
        \includegraphics[width=50pt,height=50pt]{./latexila.png}
        % Add a short explanation
        \caption{The LaTeXila icon}
    \end{center}
\end{figure}

\subsection{Tables}
\begin{table}[h]
    \begin{center}
        % Create a table with 3 left-aligned (l) columns.
        % The pipes add vertical borders
        \begin{tabular}{|l|l|l|}
            % \hline adds the horizontal lines between rows
            \hline
            % Columns are separated by an ampersand and
            % rows are terminated using the line break command \\
            Country         & Area                 & Capital\\ \hline
            Russia          & 17,075,200 $km^2$    & Moskow \\ \hline
            Canada          & 9,984,670 $km^2$     & Ottawa \\ \hline
            United States   & 9,631,418 $km^2$     & Washington, D.C. \\ \hline
            China           & 9,596,960 $km^2$     & Beijing \\ \hline
        \end{tabular}
    \end{center}
    \caption{The 4 largest countries in the world}
\end{table}



\end{document}
